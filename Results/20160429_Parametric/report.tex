\documentclass[]{article}
\usepackage{lmodern}
\usepackage{amssymb,amsmath}
\usepackage{ifxetex,ifluatex}
\usepackage{fixltx2e} % provides \textsubscript
\ifnum 0\ifxetex 1\fi\ifluatex 1\fi=0 % if pdftex
  \usepackage[T1]{fontenc}
  \usepackage[utf8]{inputenc}
\else % if luatex or xelatex
  \ifxetex
    \usepackage{mathspec}
    \usepackage{xltxtra,xunicode}
  \else
    \usepackage{fontspec}
  \fi
  \defaultfontfeatures{Mapping=tex-text,Scale=MatchLowercase}
  \newcommand{\euro}{€}
\fi
% use upquote if available, for straight quotes in verbatim environments
\IfFileExists{upquote.sty}{\usepackage{upquote}}{}
% use microtype if available
\IfFileExists{microtype.sty}{%
\usepackage{microtype}
\UseMicrotypeSet[protrusion]{basicmath} % disable protrusion for tt fonts
}{}
\usepackage[margin=1in]{geometry}
\usepackage{graphicx}
\makeatletter
\def\maxwidth{\ifdim\Gin@nat@width>\linewidth\linewidth\else\Gin@nat@width\fi}
\def\maxheight{\ifdim\Gin@nat@height>\textheight\textheight\else\Gin@nat@height\fi}
\makeatother
% Scale images if necessary, so that they will not overflow the page
% margins by default, and it is still possible to overwrite the defaults
% using explicit options in \includegraphics[width, height, ...]{}
\setkeys{Gin}{width=\maxwidth,height=\maxheight,keepaspectratio}
\ifxetex
  \usepackage[setpagesize=false, % page size defined by xetex
              unicode=false, % unicode breaks when used with xetex
              xetex]{hyperref}
\else
  \usepackage[unicode=true]{hyperref}
\fi
\hypersetup{breaklinks=true,
            bookmarks=true,
            pdfauthor={Nicolas Blöchliger, Institute of Medical Microbiology, University of Zurich},
            pdftitle={Forecast probabilities - work in progress},
            colorlinks=true,
            citecolor=blue,
            urlcolor=blue,
            linkcolor=black,
            pdfborder={0 0 0}}
\urlstyle{same}  % don't use monospace font for urls
\setlength{\parindent}{0pt}
\setlength{\parskip}{6pt plus 2pt minus 1pt}
\setlength{\emergencystretch}{3em}  % prevent overfull lines
\setcounter{secnumdepth}{5}

%%% Use protect on footnotes to avoid problems with footnotes in titles
\let\rmarkdownfootnote\footnote%
\def\footnote{\protect\rmarkdownfootnote}

%%% Change title format to be more compact
\usepackage{titling}

% Create subtitle command for use in maketitle
\newcommand{\subtitle}[1]{
  \posttitle{
    \begin{center}\large#1\end{center}
    }
}

\setlength{\droptitle}{-2em}
  \title{Forecast probabilities - work in progress}
  \pretitle{\vspace{\droptitle}\centering\huge}
  \posttitle{\par}
  \author{Nicolas Blöchliger, Institute of Medical Microbiology, University of
Zurich}
  \preauthor{\centering\large\emph}
  \postauthor{\par}
  \predate{\centering\large\emph}
  \postdate{\par}
  \date{12/05/2016}



\begin{document}

\maketitle


{
\hypersetup{linkcolor=black}
\setcounter{tocdepth}{2}
\tableofcontents
}
\section{Goals}\label{goals}

\label{sec:goals}

\begin{enumerate}
\itemsep1pt\parskip0pt\parsep0pt
\item
  Compute the probability that a strain is pseudo-WT given an observed
  diameter \(y\).
\item
  Compute the probability that a strain is susceptible according to
  official breakpoint given an observed diameter \(y\).
\end{enumerate}

\section{Model}\label{model}

We assume:

\begin{itemize}
\item The distribution of the true diameter $X$ is a mixture of three components with weights $w_i=p(C=i)$, where $C$ encodes the component. The true diameter is 6\hspace{3pt}mm for the first component and normally distributed for the other two components:
\[
p_i(x)=f_X(x|C=i)=
  \begin{cases}
    \delta_6(x)       & \quad \text{if } i=1,\\
    \phi(x;\mu_i,\sigma^2)  & \quad \text{else,}\\
  \end{cases}
\]
where
\[ \delta_6(x) =
  \begin{cases}
    \infty       & \quad \text{if } x=6\hspace{3pt}\text{mm},\\
    0  & \quad \text{else.}\\
  \end{cases}
\]
Thus,
\[
f_X(x)=w_1\delta_6+\sum_{i=2}^3w_i\phi(x;\mu_i,\sigma^2).
\]

\item We observe $Y=X+E$, where $E$ models technical error. $E$ is zero for the first component and normally distributed and independent of $X$ with mean $\mu_E=0$ and constant variance $\sigma_E^2$ for the other two components. The conditional density of $Y$ given the component $C$ is therefore
\[ f_Y(y|C=i) =
  \begin{cases}
    \delta_6(y)       & \quad \text{if } i=1,\\
    p_i\ast\phi(\hspace{1pt}\cdot\hspace{1pt};0,\sigma_E^2)=\phi(y;\mu_i,\sigma^2+\sigma_E^2)  & \quad \text{else.}\\
  \end{cases}
\]
Thus,
\[
f_Y(y)=w_1\delta_6+\sum_{i=2}^3w_i\phi(y;\mu_i,\sigma^2+\sigma_E^2).
\]
Note that we do not account for the fact that the observed data are rounded to integer values.
\end{itemize}

\subsection{Estimation of model
parameters}\label{estimation-of-model-parameters}

\begin{itemize}
\itemsep1pt\parskip0pt\parsep0pt
\item
  We estimate \(w_1\) as the fraction of data points in the sample that
  are equal to 6\hspace{3pt}mm.
\item
  The parameters of the second and third component of \(Y\),
  i.e.~\(w_i\), \(\mu_i\), and \(\sigma^2+\sigma_E^2\), are estimated by
  fitting a normal mixture model of two components to the data in the
  sample with diameters greater than 6\hspace{3pt}mm. We use the R
  package \texttt{mclust}.
\item
  Estimates for the variance of the error \(\sigma_E^2\) will be taken
  from independent work in order to obtain \(\sigma^2\).
\end{itemize}

\subsection{Limitations of the model}\label{limitations-of-the-model}

\begin{itemize}
\itemsep1pt\parskip0pt\parsep0pt
\item
  The model does not account for the fact that \(X\geq6\)\hspace{3pt}mm.
  As long as the means of the two components are sufficiently large (say
  \(\mu_i-6\hspace{3pt}\text{mm}>2\sigma\)), this should not cause
  problems.
\item
  The model does not account for the fact that
  \(Y\leq40\)\hspace{3pt}mm. As long as the means of the two components
  are sufficiently small (say
  \(40\hspace{3pt}\text{mm}-\mu_i>2\sigma\)), this should not cause
  problems.
\item
  The error is assumed to be normally distributed with constant
  variance. This assumption is obviously violated if \(X\) is close to 6
  or 40\hspace{3pt}mm. It is also violated for antibiotics like CPD, for
  which diameters are distorted in order to avoid additional laborious
  tests.
\item
  The distributions of the two components are assumed to be normal. This
  seems fine for the component corresponding to wild-type strains.
  However, the distribution of the component corresponding to the
  resistant strains with \(X>6\)\hspace{3pt}mm might not be adequately
  modelled.
\item
  The variances of the two components are assumed to be equal. This
  assumption is problematic but has the advantage of guaranteeing that
  there is only one decision boundary if strains are assigned to the
  more likely component.
\end{itemize}

\section{Data}\label{data}

\emph{E. coli}, \(\beta\)-lactams. \emph{To be completed.}

\section{Results}\label{results}

The figures in this document are organized as follows. Note that the the
first and the second component are combined for visualisation.

\begin{itemize}
\itemsep1pt\parskip0pt\parsep0pt
\item
  Top-left: Histogram of sample and the estimated density of \(Y\)
  (black) and its components (coloured). The contribution from the first
  component (\(\delta_6\)) is visualised as a uniform distribution with
  support {[}5.5\hspace{3pt}mm, 6.5\hspace{3pt}mm{]}.
\item
  Middle-left: Empirical cumulative distribution function (cdf) of \(Y\)
  (grey), its estimate (black) and estimated cdfs for the components of
  \(Y\) (coloured).
\item
  Bottom-left: \(p(C=i|Y=y)\), i.e.~the probability that a data point is
  associated with component \(i\) given an observed diameter \(y\). For
  this calculation, the first two components were grouped together.
\item
  Top-right: Q-Q plot. If the estimated density of \(Y\) explained the
  data perfectly, all point would lie on the identity line (grey).
\item
  Middle-right: Histogram of sample and the estimated density of \(X\)
  for various values of \(\sigma_E\).
\item
  Bottom-right: \(p(X\leq t|Y=y)\), i.e.~the probability that the true
  diameter is below a breakpoint \(t\) given an observed diameter \(y\).
  For the time being, \(t\) was set such that \(p(C=i|Y=t)\approx0.5\).
\end{itemize}

\pagebreak

\subsection{AM10}\label{am10}

\(\mu_{2,3}=\) 11.5 mm, 21.4 mm. \(\sqrt{\sigma^2+\sigma_E^2}=\) 3 mm.
\(w=\) 0.47, 0.09, 0.91.

\includegraphics{report_files/figure-latex/unnamed-chunk-4-1.pdf}

\pagebreak

\subsection{KF}\label{kf}

\(\mu_{2,3}=\) 13.2 mm, 19.6 mm. \(\sqrt{\sigma^2+\sigma_E^2}=\) 2.7 mm.
\(w=\) 0.12, 0.12, 0.88.

\includegraphics{report_files/figure-latex/unnamed-chunk-4-2.pdf}

\pagebreak

\subsection{FOX}\label{fox}

\(\mu_{2,3}=\) 14.1 mm, 25.5 mm. \(\sqrt{\sigma^2+\sigma_E^2}=\) 3.1 mm.
\(w=\) 0.01, 0.03, 0.97.

\includegraphics{report_files/figure-latex/unnamed-chunk-4-3.pdf}

\pagebreak

\subsection{CPD}\label{cpd}

\(\mu_{2,3}=\) 11.7 mm, 26.6 mm. \(\sqrt{\sigma^2+\sigma_E^2}=\) 3.3 mm.
\(w=\) 0.09, 0.03, 0.97.

\includegraphics{report_files/figure-latex/unnamed-chunk-4-4.pdf}

\pagebreak

\subsection{AMC}\label{amc}

\(\mu_{2,3}=\) 13.7 mm, 22.1 mm. \(\sqrt{\sigma^2+\sigma_E^2}=\) 3 mm.
\(w=\) 0.01, 0.16, 0.84.

\includegraphics{report_files/figure-latex/unnamed-chunk-4-5.pdf}

\pagebreak

\subsection{TPZ}\label{tpz}

\(\mu_{2,3}=\) 15 mm, 24.7 mm. \(\sqrt{\sigma^2+\sigma_E^2}=\) 3.2 mm.
\(w=\) 0, 0.03, 0.97.

\includegraphics{report_files/figure-latex/unnamed-chunk-4-6.pdf}

\pagebreak

\subsection{CXM}\label{cxm}

\(\mu_{2,3}=\) 14.1 mm, 24.6 mm. \(\sqrt{\sigma^2+\sigma_E^2}=\) 2.8 mm.
\(w=\) 0.09, 0.02, 0.98.

\includegraphics{report_files/figure-latex/unnamed-chunk-4-7.pdf}

\pagebreak

\subsection{CTX}\label{ctx}

\(\mu_{2,3}=\) 12.7 mm, 30 mm. \(\sqrt{\sigma^2+\sigma_E^2}=\) 3.6 mm.
\(w=\) 0.05, 0.05, 0.95.

\includegraphics{report_files/figure-latex/unnamed-chunk-4-8.pdf}

\pagebreak

\subsection{CAZ}\label{caz}

\(\mu_{2,3}=\) 13.8 mm, 27.2 mm. \(\sqrt{\sigma^2+\sigma_E^2}=\) 3.1 mm.
\(w=\) 0.02, 0.06, 0.94.

\includegraphics{report_files/figure-latex/unnamed-chunk-4-9.pdf}

\pagebreak

\subsection{CRO}\label{cro}

\(\mu_{2,3}=\) 12.1 mm, 30.8 mm. \(\sqrt{\sigma^2+\sigma_E^2}=\) 2.9 mm.
\(w=\) 0.04, 0.07, 0.93.

\includegraphics{report_files/figure-latex/unnamed-chunk-4-10.pdf}

\pagebreak

\subsection{FEP}\label{fep}

\(\mu_{2,3}=\) 16.7 mm, 31.6 mm. \(\sqrt{\sigma^2+\sigma_E^2}=\) 3.4 mm.
\(w=\) 0, 0.08, 0.92.

\includegraphics{report_files/figure-latex/unnamed-chunk-4-11.pdf}

\pagebreak

\subsection{ETP}\label{etp}

\(\mu_{2,3}=\) 30 mm, 35.3 mm. \(\sqrt{\sigma^2+\sigma_E^2}=\) 3 mm.
\(w=\) 0, 0.52, 0.48.

\includegraphics{report_files/figure-latex/unnamed-chunk-4-12.pdf}

\pagebreak

\subsection{IPM}\label{ipm}

\(\mu_{2,3}=\) 29.2 mm, 33.9 mm. \(\sqrt{\sigma^2+\sigma_E^2}=\) 2.5 mm.
\(w=\) 0.8, 0.2.

\includegraphics{report_files/figure-latex/unnamed-chunk-4-13.pdf}

\pagebreak

\subsection{MEM}\label{mem}

\(\mu_{2,3}=\) 30.8 mm, 35.1 mm. \(\sqrt{\sigma^2+\sigma_E^2}=\) 2.6 mm.
\(w=\) 0.57, 0.43.

\includegraphics{report_files/figure-latex/unnamed-chunk-4-14.pdf}

\section{Conclusion}\label{conclusion}

In discussions with Peter Keller and Michael Hombach on May, 11th and
12th, 2016, we decided to split this project according to the goals
stated in Sec.~\ref{sec:goals}. Roadmap:

\begin{itemize}
\itemsep1pt\parskip0pt\parsep0pt
\item
  Compare \(p\)(pseudo-WT\textbar{}observed diameter) with ground truth.
\item
  Compute the probability of very major errors for CBP and CBP + 2 mm.
\item
  Investigate robustness of model \(p\)(S\textbar{}observed diameter).
\item
  Extend analysis to quinolones, tetracyclines, aminoglycosides,
  cholistin, etc.~using data cleaned by Giorgia Valsesia.
\item
  Plan manuscript.
\item
  Meet with Marc Schmid regarding implementation.
\end{itemize}

\section{Appendix}\label{appendix}

\begin{figure}[htbp]
\centering
\includegraphics{report_files/figure-latex/unnamed-chunk-5-1.pdf}
\caption{\label{fig:allDistr}Distributions of diameters.}
\end{figure}

\end{document}
