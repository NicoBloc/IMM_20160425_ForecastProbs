\documentclass[a4paper]{article}
\usepackage{lmodern}
\usepackage{amssymb,amsmath}
\usepackage{ifxetex,ifluatex}
\usepackage{fixltx2e} % provides \textsubscript
\ifnum 0\ifxetex 1\fi\ifluatex 1\fi=0 % if pdftex
  \usepackage[T1]{fontenc}
  \usepackage[utf8]{inputenc}
\else % if luatex or xelatex
  \ifxetex
    \usepackage{mathspec}
    \usepackage{xltxtra,xunicode}
  \else
    \usepackage{fontspec}
  \fi
  \defaultfontfeatures{Mapping=tex-text,Scale=MatchLowercase}
  \newcommand{\euro}{€}
\fi
% use upquote if available, for straight quotes in verbatim environments
\IfFileExists{upquote.sty}{\usepackage{upquote}}{}
% use microtype if available
\IfFileExists{microtype.sty}{%
\usepackage{microtype}
\UseMicrotypeSet[protrusion]{basicmath} % disable protrusion for tt fonts
}{}
\usepackage[margin=1in]{geometry}
\usepackage{graphicx}
\makeatletter
\def\maxwidth{\ifdim\Gin@nat@width>\linewidth\linewidth\else\Gin@nat@width\fi}
\def\maxheight{\ifdim\Gin@nat@height>\textheight\textheight\else\Gin@nat@height\fi}
\makeatother
% Scale images if necessary, so that they will not overflow the page
% margins by default, and it is still possible to overwrite the defaults
% using explicit options in \includegraphics[width, height, ...]{}
\setkeys{Gin}{width=\maxwidth,height=\maxheight,keepaspectratio}
\ifxetex
  \usepackage[setpagesize=false, % page size defined by xetex
              unicode=false, % unicode breaks when used with xetex
              xetex]{hyperref}
\else
  \usepackage[unicode=true]{hyperref}
\fi
\hypersetup{breaklinks=true,
            bookmarks=true,
            pdfauthor={Nicolas Blöchliger, Institute of Medical Microbiology, University of Zurich},
            pdftitle={Forecast probabilities - modelling literature review},
            colorlinks=true,
            citecolor=blue,
            urlcolor=blue,
            linkcolor=black,
            pdfborder={0 0 0}}
\urlstyle{same}  % don't use monospace font for urls
\setlength{\parindent}{0pt}
\setlength{\parskip}{6pt plus 2pt minus 1pt}
\setlength{\emergencystretch}{3em}  % prevent overfull lines
\setcounter{secnumdepth}{5}

%%% Use protect on footnotes to avoid problems with footnotes in titles
\let\rmarkdownfootnote\footnote%
\def\footnote{\protect\rmarkdownfootnote}

%%% Change title format to be more compact
\usepackage{titling}

% Create subtitle command for use in maketitle
\newcommand{\subtitle}[1]{
  \posttitle{
    \begin{center}\large#1\end{center}
    }
}

\setlength{\droptitle}{-2em}
  \title{Forecast probabilities - modelling literature review}
  \pretitle{\vspace{\droptitle}\centering\huge}
  \posttitle{\par}
  \author{Nicolas Blöchliger, Institute of Medical Microbiology, University of
Zurich}
  \preauthor{\centering\large\emph}
  \postauthor{\par}
  \predate{\centering\large\emph}
  \postdate{\par}
  \date{01/06/2016}



\begin{document}

\maketitle


{
\hypersetup{linkcolor=black}
\setcounter{tocdepth}{2}
\tableofcontents
}
\section{Error rate-bound method
{[}1{]}}\label{error-rate-bound-method-metzler1974}

Aim: Determine CBPs for inhibition zone diameters. Required: MIC
cut-off, desired sensitivity and specificity. Model: Classification
according to this cut-off is treated as the ground truth. CBPs for
inhibition zone diameters are set such that desired sensitivity and
specificity are achieved for a minimal intermediate zone.

Ref.~{[}2{]} presents a cost function and an algorithm for its
optimisation to apply the error rate-bound method in cases with two
different MIC cut-offs.

\section{{[}3{]}}\label{craig2000}

Motivation: the error-rate bound method

\section*{Reference}\label{reference}
\addcontentsline{toc}{section}{Reference}

{[}1{]}C.M. Metzler, R.M. DeHaan, Susceptibility tests of anaerobic
bacteria: Statistical and clinical considerations, J.~Infect.~Dis. 130
(1974) 588--594.

{[}2{]}M.N. Brunden, G.E. Zurenko, B. Kapik, Modification of the
error-rate bounded classification scheme for use with two MIC break
points, Diagn.~Microbiol.~Infect.~Dis. 15 (1992) 135--140.

{[}3{]}B.A. Craig, Modeling approach to diameter breakpoint
determination, Diagn.~Microbiol.~Infect.~Dis. 36 (2000) 193--202.

\end{document}
